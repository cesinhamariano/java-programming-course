\documentclass[serif,mathserif]{beamer}
\usepackage{listings}
\usepackage{fleqn}
\usepackage[utf8]{inputenc}
\usepackage{amsmath, amsfonts, epsfig, xspace}
\usepackage{pstricks,pst-node}
\usepackage{multimedia}
\usepackage[normal,tight,center]{subfigure}
\setlength{\subfigcapskip}{-.5em}
\usepackage{beamerthemesplit}
\renewcommand\sfdefault{phv}
\renewcommand\familydefault{\sfdefault}
\usetheme{default}
\usepackage{color}
\useoutertheme{default}
\usepackage{texnansi}
\usepackage{marvosym}
\definecolor{bottomcolour}{rgb}{0.32,0.3,0.38}
\definecolor{middlecolour}{rgb}{0.08,0.08,0.16}
\setbeamerfont{title}{size=\Huge}
\setbeamercolor{structure}{fg=gray}
\setbeamertemplate{frametitle}[default]%[center]
\setbeamercolor{normal text}{bg=black, fg=white}
\setbeamertemplate{background canvas}[vertical shading]
[bottom=bottomcolour, middle=middlecolour, top=black]
\setbeamertemplate{items}[circle]
\setbeamerfont{frametitle}{size=\huge}
\setbeamertemplate{navigation symbols}{} %no nav symbols
\lstset{language=Java}


\author[Vinicius Miana]{Vinicius Miana}

\title[Short Title\hspace{2em}\insertframenumber/\inserttotalframenumber]{Programação Defensiva}

%\date{17 de abril de 2013} %leave out for today's date to be insterted

\institute{Universidade Presbiteriana Mackenzie}

\begin{document}

\maketitle

\section{Roteiro da Apresentação}  

\begin{frame}
  \frametitle{Conteúdo}
  Tópicos abordados:
  \begin{itemize}
  \item Um primeiro exemplo;
  \item Definição;
  \item Principais técnicas;
  \item Exemplos, exemplos e mais exemplos.
  \end{itemize}
\end{frame}


\begin{frame}[fragile]
  \frametitle{Primeiro Exemplo}
  Exemplo
\begin{lstlisting}
Aluno a = alunoDAO.getAluno(id);
addCourse(a.getId(),a.getName(),"Programacao Java");
\end{lstlisting}
e se a for null??
\begin{lstlisting}
Aluno a = alunoDAO.getAluno(id);
if(a != null) {
     addCourse(a.getId(),a.getName(),"Programacao Java");
}
\end{lstlisting}
desta maneira evitamos um NPE. 
\end{frame}

\begin{frame}[fragile]
  \frametitle{Definição}
  Exemplo
\begin{lstlisting}
Aluno a = alunoDAO.getAluno(id);
addCourse(a.getId(),a.getName(),"Programacao Java");
\end{lstlisting}
e se a for null??
\begin{lstlisting}
Aluno a = alunoDAO.getAluno(id);
if(a != null) {
     addCourse(a.getId(),a.getName(),"Programacao Java");
}
\end{lstlisting}
desta maneira evitamos um NPE. 
\end{frame}



\begin{frame}
  \frametitle{Questions}
\end{frame}
\end{document}

